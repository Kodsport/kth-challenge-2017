\problemname{Lights Out}

\illustration{0.5}{bulb}{\href{https://pixabay.com/en/light-bulb-current-light-glow-503881/}{Picture} by Comfreak on pixabay, cc0}%
\noindent
Poor Eve, she is almost always the last one at work. Most
unfortunately, she is very afraid of the dark, and the company rules
dictates the last one leaving the office is obliged to make sure all
the lamps at the whole office are off.  Her sloppy colleagues
sometimes forget to turn their lamps off, more often then not to be
honest, but to be fair it is not as trivial as you might think.  You
see, at Eve's workplace, each room has exactly one lamp but may have
several light switches. The thing is though, that unlike a traditional
light switch, these switches toggle a subset of all the lamps at the
office. Each switch inverts the light status, either from lit to
turned off or the other way around, for a fixed set of lamps depending
on the switch. It may even be that the lamp in the room is not
effected by any of its switches, or that there are no switches in the
room at all.

Eve is a creature of habit and wants to take a fixed route out of the
office each evening.  At the same time, the set of lamps to turn off
may be different on different days, so she has to plan for all
eventualities.  In other words, Eve wants a route through the office
such that, for any possible configuration of the lamps, there is some
combination of the light switches in the rooms Eve moves through that
will allow her to turn off that configuration of lamps.

Note that it may be the case that some configurations of lamps are
impossible to turn off (for instance, in Sample Input 1, it would be
impossible to turn off only lamp $0$), but Eve doesn't have to worry
about such configurations of lamps (because they are also impossible
to turn on).  The route only has to let her turn off configurations
which are actually possible to turn off.  Eve is fine with passing
through a room even if its lamp is turned off, for instance it is OK
if the last lamp in the office is turned off somewhere in the middle
of the route even though it means walking through a few unlit rooms at
the end.

\section*{Input}

The first line of input contains three positive integers, $n$, $m$, and $l$ ($2 \leq n \le 20$, $1 \le m \le 190$, $1 \le l \le 100$), where $n$ is the number of rooms in the office,  $m$ is the number of connections between rooms, and $l$ is the number of switches.
Next follow $m$ lines, each describing a pair of adjacent rooms containing two room numbers $a \ne b$, meaning you can enter room $b$ from room $a$ and vice versa.  No unordered pair $\{a,b\}$ appears more than once.

Then follow $l$ lines, each describing a light switch. Each such line
starts with a room number telling which room the light switch is
in. The second integer on the line $p>0$ gives the number of lamps
that are toggled by the switch. The remainder of the line contains $p$
room numbers. You can assume that no two room identifiers are
identical in a switch's toggle list.

The rooms are numbered $0$ through $n-1$. The room
from which Eve leave's the office is number $0$, and Eve's room (where
she starts) is number $1$.  You may assume that it is possible to reach any room from room $1$.

\section*{Output}

Output one line with the minimum number of rooms on a path from Eve's room to the entrance room (counting both endpoints) including a set of switches making it possible to turn off any possible subset of lamps lit. Note that she might need to visit a room multiple times, in which case each visit should be counted.

%%% Local Variables:
%%% mode: latex
%%% TeX-master: "../../challenge-2017"
%%% End:
